% easychair.tex,v 3.5 2017/03/15

\documentclass{easychair}
%\documentclass[EPiC]{easychair}
%\documentclass[EPiCempty]{easychair}
%\documentclass[debug]{easychair}
%\documentclass[verbose]{easychair}
%\documentclass[notimes]{easychair}
%\documentclass[withtimes]{easychair}
%\documentclass[a4paper]{easychair}
%\documentclass[letterpaper]{easychair}

\usepackage{babel}
\usepackage[utf8]{inputenc}
\usepackage{geometry}
\usepackage{doc}
\usepackage{xspace}

% use this if you have a long article and want to create an index
% \usepackage{makeidx}

% In order to save space or manage large tables or figures in a
% landcape-like text, you can use the rotating and pdflscape
% packages. Uncomment the desired from the below.
%
% \usepackage{rotating}
% \usepackage{pdflscape}

% Some of our commands for this guide.
%
\newcommand{\easychair}{\textsf{easychair}}
\newcommand{\miktex}{MiK{\TeX}}
\newcommand{\texniccenter}{{\TeX}nicCenter}
\newcommand{\makefile}{\texttt{Makefile}}
\newcommand{\latexeditor}{LEd}

\newcommand{\BBtheFourth}{107}
\newcommand{\SigmaTheFourth}{13}
\newcommand{\BBtheFourthTNF}{858{,}909}

\newcommand{\BBtheFifth}{47{,}176{,}870}
\newcommand{\BBtheFifthTNF}{181{,}385{,}789}
\newcommand{\SigmaTheFifth}{4{,}098}

\newcommand{\BBTxF}{3{,}932{,}974}
\newcommand{\BBTxFTNF}{2{,}154{,}217}

\newcommand{\radofull}{Tibor Rad\'o\xspace}
\newcommand{\rado}{Rad\'o\xspace}

\newcommand{\cryptid}{Cryptid\xspace}

\newcommand{\cycler}{Cycler\xspace}
\newcommand{\cyclers}{Cyclers\xspace}
\newcommand{\TC}{Translated Cycler\xspace}
\newcommand{\TCs}{Translated Cyclers\xspace}

\newcommand{\headpos}{head-position\xspace}
\newcommand{\headposs}{head-positions\xspace}

\newcommand{\states}{\mathcal{S}}
\newcommand{\alphabet}{\mathcal{A}}
\newcommand{\symbolzero}{\texttt{0}}
\newcommand{\symbolone}{\texttt{1}}

\newcommand{\numSporadic}{\ts{XX}\xspace}

\newcommand{\ssp}{state-symbol-pair\xspace}
\newcommand{\ssps}{state-symbol-pairs\xspace}

\newcommand{\HALT}{\texttt{HALT}\xspace}
\newcommand{\NONHALT}{\texttt{NONHALT}\xspace}
\newcommand{\UNKNOWN}{\texttt{UNKNOWN}\xspace}

\newcommand{\CoqBB}{Coq-BB5\xspace}
\newcommand{\CoqBBnospace}{Coq-BB5}

\usepackage{amsmath,amsfonts,amssymb,amsthm}

\theoremstyle{definition} % don't use italics
\newtheorem{theorem}{Theorem}[section]
\newtheorem{definition}{Definition}[section]
\newtheorem{lemma}{Lemma}[section]
\newtheorem{proposition}{Proposition}[section]
\newtheorem{corollary}{Corollary}[section]
\numberwithin{equation}{section}


\theoremstyle{definition} % emphasize the "Remark N." with italics, not bold
\newtheorem{observation}{Observation}[section]
\newtheorem{example}{Example}[section]
\newtheorem{remark}{Remark}[section]
\newtheorem*{theorem*}{Theorem}

% \usepackage[shortlabels]{enumitem}
% \setlist[enumerate]{nosep}

%\makeindex

%% Front Matter
%%
% Regular title as in the article class.
%

\title{
  \vspace{-1em}  
Coq proof of the fifth Busy Beaver value}

% Authors are joined by \and. Their affiliations are given by \inst, which indexes
% into the list defined using \institute
%
\author{
  Tristan Stérin\thanks{Presenting author}\inst{~1,2} \and The bbchallenge Collaboration\inst{1}
}

% Institutes for affiliations are also joined by \and,
\institute{
  bbchallenge.org, see credits,
   \email{bbchallenge@bbchallenge.org}
\and
prgm.dev,
  Paris, France
  \email{tristan@prgm.dev}\\
 }

%  \authorrunning{} has to be set for the shorter version of the authors' names;
% otherwise a warning will be rendered in the running heads. When processed by
% EasyChair, this command is mandatory: a document without \authorrunning
% will be rejected by EasyChair

\authorrunning{bbchallenge}

% \titlerunning{} has to be set to either the main title or its shorter
% version for the running heads. When processed by
% EasyChair, this command is mandatory: a document without \titlerunning
% will be rejected by EasyChair
\titlerunning{Coq proof of the fifth Busy Beaver value}




\begin{document}


\maketitle
\vspace{-1em}
\begin{abstract}
  We prove that $S(5) = 47,176,870$. The Busy Beaver value $S(n)$ is the maximum~number of steps a halting n-state 2-symbol Turing machine can perform from the all-0 tape before halting and $S$ was historically introduced as one of the simplest examples of a noncomputable function.  Using the Coq proof assistant, we enumerate 181,385,789 5-state Turing machines, and for each, decide whether it halts or not. 
  Our result marks the first determination of a new Busy Beaver value in over 40 years, leveraging Coq's computing capabilities and demonstrating the effectiveness of collaborative online research.
\end{abstract}
\newcommand{\ie}{i.e.\ }
\newcommand{\eg}{e.g.\ }

\newcommand{\noncomput}{noncomputable\xspace}
\newcommand{\BBfull}{Busy Beaver\xspace}
\newcommand{\Coq}{Coq\xspace}
% We'll update this link to v1 release URL
\newcommand{\CoqProofReleaseURL}{\url{https://github.com/ccz181078/Coq-BB5}}


Introduced by \radofull in 1962 as part of the \textit{Busy Beaver game} \cite{Rado_1962}, $S(n)$ is the maximum number of steps that a halting $n$-state 2-symbol Turing machine\footnote{Turing machines with one bi-infinite tape and allowing undefined transitions.} can do from the all-0 tape before halting. The function $S$ is noncomputable as it would otherwise allow to solve the 
halting problem: run an $n$-state machine for $S(n)+1$ steps and, if it has not halted by then, you know it will never halt.

As $S$ in noncomputable, there is no algorithm that computes \textit{all} of its values, but one can still try to determine \textit{some} of them. We can get a sense of how hard this is by noting that, for instance, there is a 25-state 2-symbol Turing machine \cite{GoldbachTM27,GoldbachTM25} that iterates all even natural numbers in search of a counterexample to Goldbach's conjecture\footnote{The conjecture states that every even natural number $n>2$ can be written as the sum of two primes. This is one of the oldest open problems in mathematics.}, and halts iff one is found (\ie iff the conjecture is false), which means that proving the value of $S(\geq 25)$ is at least as hard as proving or disproving Goldbach's conjecture. Worse, this idea can be generalised to any other $\Pi_1$ statement\footnote{\ie a logical statement of the form ``For all $x$, $\phi(x)$'' with $\phi$ using only bounded quantifiers meaning that a computer can check $\phi(x)$ in finite time for any $x$.}, such as Riemann Hypothesis or ZF's consistency which have been respectively reduced to $S(744)$ and $S(748)$ \cite{RiemannTM,Yedidia2016,ZFTM,Yedidia2016,BusyBeaverFrontier,BB748Thesis}. 
%Aaronson conjectures that as low as $S(10)$ is independent of Peano \cite{BusyBeaverFrontier}.

Nonetheless, researchers have embarked on the quest of finding what is the biggest value of $S$ we can know. Motivation include: (i) trying to find the simplest open problem in mathematics on the Busy Beaver scale by exhibiting an $n$-state Turing machine for which it is unknown whether it halts or not but $S(n-1)$ is known (\ie the halting status of all $n-1$-state Turing machines is known) and (ii) studying compute \textit{in the wild} by having to understand (to some extent) the behavior of small, not human-engineered, Turing machines which often implement \textit{quirky} algorithms in a \textit{quirky} way.


Prior to this work, only the four first values of $S$ had been proved: $S(1)=1$, $S(2)=6$ \cite{Rado_1962}, $S(3) = 21$ \cite{Lin1963}, $S(4) = 107$ \cite{Brady83}. After some early attempts in the 1960s and 1970s~\cite{PMichel_website,michel2019busy}, the search for $S(5)$ took a turn in 1989 when Marxen and Buntrock found a new 5-state champion\footnote{\url{https://bbchallenge.org/1RB1LC_1RC1RB_1RD0LE_1LA1LD_1RZ0LA}} (\ie a 5-state machine with bigger step-count than previously known ones) achieving $\BBtheFifth$ steps \cite{Marxen_1990}, thus showing $S(5) \geq \BBtheFifth$. In 2020, after thirty years without improvements, Aaronson conjectured that there was no better 5-state machine, \ie $S(5) = \BBtheFifth$ \cite{BusyBeaverFrontier}.

Our main result is to prove this conjecture, using the \Coq proof assistant\footnote{The \Coq proof assistant has been renamed Rocq.} \cite{Coq8.20}:

\begin{theorem*}[\texttt{Lemma BB5\_value}]
  $S(5) = \BBtheFifth$.
\end{theorem*}

The proof, called \CoqBB, is available at \url{https://github.com/ccz181078/Coq-BB5}.

\begin{table}[h!]
  \centering
  \begin{tabular}{|l|rrr|}
    \hline
    $S(5)$ decider pipeline                                & Nonhalt                         & Halt                           & Total decided \\
    \hline
    1. Loops                                               & 126,994,099                     & 48,379,711                     & 175,373,810   \\
    2. $n$-gram Closed Position Set (NGramCPS)             & 6,005,142                       & 0                              & 6,005,142     \\
    3. Repeated Word List (RepWL)                          & 6,577                           & 0                              & 6,577         \\
    4. Finite Automata Reduction (FAR)                     & 23                              & 0                              & 23            \\
    5. Weighted Finite Automata Reduction (WFAR)           & 17                              & 0                              & 17            \\
    
    6. Long halters (simulation up to $\BBtheFifth$ steps) & 0                               & 183                            & 183           \\
    7. Sporadic Machines, individual proofs                & 13                              & 0                              & 13            \\
    8. Reduction to \texttt{1RB}                           & 14                              & 0                              & 14            \\ \hline
    Total                                                  & \multicolumn{1}{r}{133,005,895} & \multicolumn{1}{r}{48,379,894} & 181,385,789   \\ \hline
  \end{tabular}
  \caption{Approximation of the $S(5)$ decider pipeline as implemented in \CoqBB. All the $\BBtheFifthTNF$ enumerated 5-state machines are decided by this pipeline, which solves $S(5) = \BBtheFifth$. }\label{tab:pipelineBB5}
\end{table}

\newpage

In this talk, we will present \CoqBB and describe, at various technical levels, how the proof works. The overall structure of the proof is as follows: the proof enumerates 5-state machines in \textit{Tree Normal Form} (\textbf{TNF}) \cite{Brady64,Brady83,Marxen_1990}. TNF essentially consists in enumerating partially-defined Turing machines starting from a machine with no transitions defined, each enumerated machine is passed through a pipeline of \textit{deciders} which are algorithms trying to decide whether the machine halts or not
\footnote{We don't use the word ``decider'' in the usual sense of theoretical computer science, formally we mean an algorithm that (i) takes as input a Turing machine, (ii) is guaranteed to finish, and (iii) returns either \texttt{HALT}, \texttt{NONHALT} or \texttt{UNKNOWN}. Proving that a decider is correct means that, if the decider returns \texttt{HALT}/\texttt{NONHALT} then, indeed, the machine halts/does not halt.}:
\begin{enumerate}
  \item If the machine halts, \ie meets an undefined transition, a new subtree of machines is visited for all the possible ways to fill the undefined transition.
  \item If the machine does not halt, it is a leaf of the TNF tree.
\end{enumerate}
The TNF enumeration terminates when all leafs have been reached, \ie all the enumerated Turing machines have been decided and there are no more halting machines to expand into subtrees. The proof enumerates $181{,}385{,}789$ machines\footnote{The list of enumerated machines, with halting status and decider information is made available at: \url{https://docs.bbchallenge.org/CoqBB5_release_v1.0.0/}.} in about 45 minutes\footnote{When compiled in parallel using \texttt{native\_compute}. Unparallelised compilation with \texttt{vm\_compute} takes about 13 hours (on a Macbook Pro M3 Max).}, and runs the pipeline of deciders summarised in Table~\ref{tab:pipelineBB5} on each of them. All the algorithms (TNF enumeration and deciders) are programmed and proved correct in Coq.

The deciders essentially leave 13 \textit{Sporadic Machines} undecided, for which we provide individual Coq proof of correctness \cite{busycoq,xu2024skelet17fifthbusy}. Among them, the monstruous \textit{Skelet \#1}\footnote{\url{https://bbchallenge.org/1RB1RD_1LC0RC_1RA1LD_0RE0LB_---1RC}}, which is a translated loop (\ie eventually repeats the same pattern translated in space) that only starts looping after $10^{51}$ steps with a period of more than $8$ billion steps \cite{ShawnSkelet1}. This machine is named after Georgi Georgiev (a.k.a Skelet) who first found this machine, as well as 42 other arduous ones, back in 2003 \cite{Skelet_bbfind}.

The talk will also discuss meta aspects of this result such as how were proof assistants used in practice and how the research was conducted within The bbchallenge Collaboration, a self-organised, international, online community of scientists (mostly not working in academia) gathered around the \url{bbchallenge.org} platform \cite{sterin_2022_14955828}.

\newpage

% in general with the following observation: any $\Pi_1$ statement, that is, a logical statement of the form ``For all $x$, $\phi(x)$'' with $\phi$ using only bounded quantifiers\footnote{Hence, a program can check $\phi(x)$ in finite time for any $x$.}, can be encoded as the halting problem of a Turing machine such that the statement is false if and only if the machine halts from all-0 tape. Indeed, it is enough to enumerate each $x$, test $\phi(x)$ on it and, if it does not hold, a counterexample is found and we can halt.


\bibliographystyle{abbrv}
\newpage
\paragraph{The bbchallenge Collaboration (credits).} Here we list the main contributions that resulted in the $S(5)$ proof: mxdys (\CoqBB); Nathan Fenner, Georgi Georgiev, savask (NGramCPS); Justin Blanchard, Mateusz Naściszewski, Konrad Deka (FAR); Iijil (WFAR); mei (\texttt{busycoq}); Shawn Ligocki, mei, Jason Yuen (Shift Overflow Counters); Shawn Ligocki, Pavel Kropitz, mei (Skelet \#1); Chris Xu, mxdys (Skelet \#17); Shawn Ligocki, Dan Briggs, mei (Skelet \#10); Justin Blanchard, mei (Finned machines); Yannick Forster (Coq proof review and optimisation); Tristan Stérin (bbchallenge.org).
\bibliography{abstract}

\end{document}

