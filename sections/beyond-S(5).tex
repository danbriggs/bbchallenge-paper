% !TeX root = ../bbchallenge-paper.tex

A law of Busy Beaver study is that those who prove $BB(n) = x$ claim it is impossible to do the same for $BB(n+1)$. Despite this ignominious history, we cautiously posit $BB(6)$ will be extremely hard, if not impossible, to prove. Unlike previous investigators, we base our claim not on the strength of modern computation and the galatic size of large TMs (the current $BB(6)$ champion runs for far, far more steps than there are atoms in the universe), but on the mathematical hardness of the machines that remain.
We call mathematically difficult TMs \textit{cryptids}, and will discuss them later in this section. We know of cryptids in every part of the Busy Beaver frontier $BB(2,5)$, $BB(3,3)$, and $BB(6,2) = BB(6)$. Our canonical example is ``Antihydra," a machine which encodes an orbit under an operation very similar to that of the infamous Collatz Conjecture.

\begin{example}
    ``Antihydra'' \\
    ``Antihydra" \tm{1RB1RA_0LC1LE_1LD1LC_1LA0LB_1LF1RE_---0RA} \\
    Let $f(n) = n + \lfloor \frac{n}{2} \rfloor$, or, 
\end{example}
