\section{Enumerating Turing machines in Tree Normal Form (TNF)}\label{sec:enum}
\vspace{-10pt}
% Define a macro for the table with coloring
\begin{figure}[ht]
    \centering
    \resizebox{\textwidth}{!}{ % Rescale to fit horizontally
        \begin{tikzpicture}[
                level distance=45mm, % Increased vertical spacing
                sibling distance=70mm, % Slightly increase horizontal spacing
                every node/.style={align=center},
                edge from parent/.style={draw, -latex, thick}
            ]

            % Define a macro for the table
            \newcommand{\turingTable}[2]{
                \adjustbox{valign=t}{
                    \begin{tabular}{ccc}
                        \toprule
                                            & \textbf{0} & \textbf{1} \\
                        \midrule
                        {\color{red} A}     & #1         & #2         \\
                        {\color{orange} B}  & ---        & ---        \\
                        {\color{blue} C}    & ---        & ---        \\
                        {\color{green} D}   & ---        & ---        \\
                        {\color{magenta} E} & ---        & ---        \\
                        \bottomrule
                    \end{tabular}
                }
            }

            % Root node
            \node (root) at (0,0) {\turingTable{---}{---}};

            % First-level children (Ensuring proper spacing)
            \node (child1) at (-6,-4) {\turingTable{0R\stateA}{---}};
            \node (child2) at (-2,-4) {\turingTable{1R\stateA}{---}};
            \node (child3) at (2,-4) {\turingTable{0R\stateB}{---}};
            \node (child4) at (6,-4) {\turingTable{1R\stateB}{---}};

            % Straight downward arrows
            \draw[-latex, thick] (root.south) -- (child1.north);
            \draw[-latex, thick] (root.south) -- (child2.north);
            \draw[-latex, thick] (root.south) -- (child3.north);
            \draw[-latex, thick] (root.south) -- (child4.north);

            % Highlight the first transition (0, A) in the root table
            \draw[magenta, thick, line width=0.8mm]
            ($(root.north west) + (0.9cm,-1.25cm)$)
            rectangle
            ($(root.north west) + (1.7cm,-0.85cm)$);

            % Highlight the B0 transition in the child3 table
            \draw[magenta, thick, line width=0.8mm]
            ($(child3.north west) + (1.08cm,-1.67cm)$)
            rectangle
            ($(child3.north west) + (1.88cm,-1.27cm)$);

            % Highlight the B0 transition in the child4 table
            \draw[magenta, thick, line width=0.8mm]
            ($(child4.north west) + (1.08cm,-1.67cm)$)
            rectangle
            ($(child4.north west) + (1.88cm,-1.27cm)$);

            \node at ($(root.north) + (0,0.1cm)$) {\textbf{TNF Root}};

            % Add "Does not halt!" below child1 and child2
            \node at ($(child1.south) + (0,-0.3cm)$) {\textbf{\textit{Does not halt!}}};
            \node at ($(child2.south) + (0,-0.3cm)$) {\textbf{\textit{Does not halt!}}};

            % Fan of 12 dashed lines for child3 (wider)
            \foreach \i in {240, 246, 252, 258, 264, 270, 276, 282, 288, 294, 300} {
                    \draw[dashed, thick] ($(child3.south) + (0,0.1cm)$) -- ++(\i:1.5cm);
                }

            % Fan of 12 dashed lines for child4 (wider)
            \foreach \i in {240, 246, 252, 258, 264, 270, 276, 282, 288, 294, 300} {
                    \draw[dashed, thick] ($(child4.south) + (0,0.1cm)$) -- ++(\i:1.5cm);
                }

            % Add "12 children" below child3 and child4
            \node at ($(child3.south) + (0,-1.7cm)$) {\textbf{\textit{12 children}}};
            \node at ($(child4.south) + (0,-1.7cm)$) {\textbf{\textit{12 children}}};

            % Add "12 children" below child3 and child4
            \node at ($(child1.south) + (0,-0.9cm)$) {\textbf{\textit{No children}}};
            \node at ($(child2.south) + (0,-0.9cm)$) {\textbf{\textit{No children}}};



        \end{tikzpicture}
    }
    \caption{First-level children of the Tree Normal Form (TNF) enumeration of 5-state 2-symbol Turing machines: each node is a Turing machine, nonhalting machines are leaves of the tree. Internal nodes are halting machines, \ie machines eventually reaching an undefined transition (highlighted in magenta) and their children correspond to all the ways to define this undefined transition. By symmetry, at the first level of the TNF tree, we can ignore machines taking a left move. The two halting machines at the first-level of the tree have 12 children each corresponding to all the choices in $\{\szero,\sone\}\times\{\text{R},\text{L}\}\times\{\text{\stateA},\text{\stateB},\text{\stateC}\}$ for defining the magenta transition. Note that, in this case, the choice of states is reduced from $\{\text{\stateA},\text{\stateB},\text{\stateC},\text{\stateD},\text{\stateE}\}$ to $\{\text{\stateA},\text{\stateB},\text{\stateC}\}$ in order to prevent constructing machines that are only a permutation of one-another.}\label{fig:TNF}
\end{figure}



Syntactically, as defined in Section~\ref{sec:TMs}, there are $(2ns + 1)^{ns}$ Turing machines with $n$ states and $s$ symbols. This gives $21^{10} \simeq 1.67\times10^{13} \simeq 16 \text{ trillion}$ possible 5-state 2-symbol Turing machines. However, naively counting Turing machines that way does not account for two phenomenons:
\begin{itemize}
    \item \textbf{Unreachable transitions.} Take the 5-state 2-symbol machine where only the first transition is defined as \texttt{0RA} -- leftmost machine in Figure~\ref{fig:TNF}. This machine is the archetypal Turing machine equivalent of a ``while True'' infinite loop: the machine will never leave the transition, indefinitely drifting to the right of the tape. Hence, none of the $21^9$ machines obtained by defining the other 9 transitions are relevant since these transitions are never reached.
    \item \textbf{State/symbol permutations.} Permuting non-\stateA states and non-zero symbols (\stateA and 0 are special because the initial configuration is all-0 tape in state \stateA) creates identical machines up to renaming, hence studying the halting of only one of them is enough. State/symbol permutation divides the syntactic space size by a factor $(n-1)! (s-1)!$.
\end{itemize}

Tree Normal Form (TNF) enumeration, introduced by Brady in 1963 in his PhD thesis \cite{Brady64} and illustrated in Figure~\ref{fig:TNF} solves both of these problems: Turing machines are recursively \textit{discovered} starting from the machine with no transitions defined (TNF root). Each enumerated machine is processed by our pipeline of deciders (see Section~\ref{sec:deciders}) which will either output \HALT, \NONHALT or \UNKNOWN for each machine:
\begin{itemize}
    \item \HALT. If the machine halts, such as the rightmost machine in Figure~\ref{fig:TNF}, it means that it has met an undefined transition and children of the machine correspond to all the possible ways of defining that undefined transition (highlighted in magenta in Figure~\ref{fig:TNF}). Avoiding redundant state/symbol permutations is dealt with at this point by imposing an order on the yet-to-be-seen states/symbols, \eg children of the rightmost machine in Figure~\ref{fig:TNF} will choose between states $\{\text{\stateA},\text{\stateB},\text{\stateC}\}$ instead of $\{\text{\stateA},\text{\stateB},\text{\stateC},\text{\stateD},\text{\stateE}\}$ since $\text{\stateC}$ is the next unseen state.
    \item \NONHALT. If the machine does not halt, all its remaining undefined transitions are unreachable and the machine is a leaf of the TNF tree.
    \item \UNKNOWN. If the halting status of a machine, it is put in the basket of \textit{holdouts}, \ie machines that remain to be decided. Having solved $S(5)$ means that there are no more 5-state holdouts.
\end{itemize}

Hence, by design, TNF enumeration avoids machines with unreachable transitions and state/symbol permutations. One further optimization in the TNF algorithm is,
at the first level of the TNF tree (see Figure~\ref{fig:TNF}), to avoid machines that make a first move to the left as they can be symmetrised to go to the right instead, \eg for 5-state 2-symbol machines, this makes the TNF root have 4 children instead of 8. In the case of 2-symbol machines, it is also known that only considering machines that first write a 0 (or that first write a 1) is enough to conclude on the value $S$, but this is not used in our work \cite{Marxen_1990,busycoq}.

\begin{table}[h!]
    \centering
    \begin{tabular}{|l|r|r|r|l|}
        \hline
        $S(n)$ & \makecell{TNF nonhalting                                                          \\ machines} & \makecell{TNF halting \\ machines} & TNF Total         & Syntactic/TNF ratio                 \\
        \hline
        $S(2)$ & 42                       & 19         & 61          & TNF is 107 times smaller    \\
        $S(3)$ & 3,645                    & 1,772      & 5,417       & TNF is 891 times smaller    \\
        $S(4)$ & 609,216                  & 249,693    & 858,909     & TNF is 8,121 times smaller  \\
        $S(5)$ & 133,005,895              & 48,379,894 & 181,385,789 & TNF is 91,958 times smaller \\
        % \hline
        % \vspace{-4mm}                                                                                               \\ % Add vertical spacing for visual separation
        % \hline
        % $S(2,4)$ quasi-TNF & 1,432,880                & 721,337    & 2,154,217   & quasi-TNF is 3,238 times smaller \\
        \hline
    \end{tabular}

    \caption{TNF metrics for $S(2),\dots,S(5)$: number of leafs in the TNF tree (nonhalting machines), number of internal nodes (halting machines), total number of TNF nodes and ratio between $(4n+1)^{2n}$ which is the syntactic number of $n$-state 2-symbol machines and the number of machines in the TNF enumeration.}\label{tab:TNF-numbers}
\end{table}

TNF is unreasonably effective, as shown on Table~\ref{tab:TNF-numbers}. In the case of 5-state 2-symbol Turing machines, the total number of machines in the TNF enumeration is $\BBtheFifthTNF$, which is $91{,}958$ times smaller than $16$ trillion, the number of synctatically correct machines.

\paragraph{\CoqBB TNF implementation.} TNF enumeration, as described here, is implemented in \CoqBB, see file \texttt{TNF.v}. A \texttt{SearchQueue} abstraction with DFS capabilities is implemented, see function \texttt{SearchQueue\_upds}. The search queue is initialised with the TNF root (this is most obvious for $S(<5)$, \eg see file \texttt{BB4\_TNF\_Enumeration.v}) and deciders (see Section~\ref{sec:deciders}) are run on the enumerated Turing machines. Halting machines' children are added to the queue: the goal of the proof is to empty the queue. Taking advantage of the tree structure, compilation of the $S(5)$ proof was paralellised by isolating the 12 children of the rightmost machine in Figure~\ref{fig:TNF} in separate, independent files, see folder \texttt{BB5\_TNF\_Enumeration\_Roots/}. Parallelising the compilation made the proof compile in 3 hours (on 13 cores) instead of 13 hours. Switching to \Coq's more powerful \texttt{native\_compute} engine \cite{nativecompute} further brought parallel compilation time down to 45 minutes. This compilation time could be improved quasi-arbitrarily by splitting the tree in even more files.


% Explain algo
% talk Coq-BB5 and parallelisation, quasi TNF for bb2x4, give number of enumerated machines for S(2 - 5)

\paragraph{TNF Normalisation.} %order states in the order they are visited and L/R