\subsection{Decider pipelines for $S(5)$, $S(2,4)$ and $S(4)$}\label{sec:pipelines}
A \textit{pipeline} of deciders consists of applying several deciders in sequence: a machine is tested by each decider successively until one decider outputs \HALT or \NONHALT. Table~\ref{tab:pipelineBB5} gives an approximation of the pipeline of deciders implemented in \CoqBB in order to decide all the $\BBtheFifthTNF$ enumerated 5-state machines (see Section~\ref{sec:enum}) and that way, conclude that $S(5) = \BBtheFifth$ since any machine running more steps is decided \NONHALT, see Theorem~\ref{th:BB5}. Similarly, Table~\ref{tab:pipelineBB2x4} and Table~\ref{tab:pipelineBB4} repsectively approximations of the pipelines leading to $S(2,4) = \BBTxF$ and $S(4) = \BBtheFourth$ -- the latter confirming the result for $S(4)$ given in \cite{Brady83}.

The exact pipelines are give in Appendix~\ref{app:pipelines} and only differ in that the precise parameters and sometimes algorithmic variants are given for each decider which are sometimes interleaved with each other (\eg the decider for loops is first called with a small step-count parameter, then other deciders are applied and later on it is called again with higher step-count parameter).

\ts{TS: I'm hesitating to say this early that a subtlelty of the S(5) pipeline is that parameters (or, more "concerning" certificates) are given hardocded for approx 8k machines. Also questionning to say (or repeat) here that WFAR is irregular (as well as some sporadic), which makes a disctinction between S(5) and S(4)/S(2,4).}

\begin{table}[h!]
    \centering
    \begin{tabular}{|l|rrr|}
        \hline
        $S(5)$ decider pipeline                                                              & Nonhalt                         & Halt                           & Total decided \\
        \hline
        1. Loops, see \textbf{Section~\ref{sec:loops}}                                       & 126,994,099                     & 48,379,711                     & 175,373,810   \\
        2. $n$-gram Closed Position Set (NGramCPS), see \textbf{Section~\ref{sec:n-gramCPS}} & 6,005,142                       & 0                              & 6,005,142     \\
        3. Repeated Word List (RepWL), see \textbf{Section~\ref{sec:RepWL}}                  & 6,577                           & 0                              & 6,577         \\
        4. Finite Automata Reduction (FAR), see \textbf{Section~\ref{sec:FAR}}               & 23                              & 0                              & 23            \\
        5. Weighted Finite Automata Reduction (WFAR), see \textbf{Section~\ref{sec:WFAR}}    & 17                              & 0                              & 17            \\
        6. Long halters (simulation up to $\BBtheFifth$ steps)                               & 0                               & 183                            & 183           \\
        7. Sporadic machines, individual proofs, see \textbf{Section~\ref{sec:sporadic}}     & 13                              & 0                              & 13            \\
        8. Reduction to \texttt{1RB}, see \textbf{Section~\ref{sec:1RB_reduction}}           & 14                              & 0                              & 14            \\ \hline
        Total                                                                                & \multicolumn{1}{r}{133,005,895} & \multicolumn{1}{r}{48,379,894} & 181,385,789   \\ \hline
    \end{tabular}
    \caption{Approximation of the $S(5)$ decider pipeline as implemented in \CoqBB. All the $\BBtheFifthTNF$ enumerated 5-state machines are decided by this pipeline, which solves $S(5) = \BBtheFifth$, see Theorem~\ref{th:BB5}. The exact pipeline, with deciders parameters is given in Appendix~\ref{app:pipelines}. }\label{tab:pipelineBB5}
\end{table}

\begin{table}[h!]
    \centering
    \begin{tabular}{|l|rrr|}
        \hline
        $S(2,4)$ decider pipeline                                                            & Nonhalt   & Halt    & Total decided \\
        \hline
        1. Loops, see \textbf{Section~\ref{sec:loops}}                                       & 1,263,302 & 721,313 & 1,984,615     \\
        2. $n$-gram Closed Position Set (NGramCPS), see \textbf{Section~\ref{sec:n-gramCPS}} & 163,500   & 0       & 163,500       \\
        3. Repeated Word List (RepWL), see \textbf{Section~\ref{sec:RepWL}}                  & 6,078     & 0       & 6,078         \\
        4. Long halters (simulation up to $\BBTxF$ steps)                                    & 0         & 24      & 24            \\
        \hline
        Total                                                                                & 1,432,880 & 721,337 & 2,154,217     \\ \hline
    \end{tabular}
    \caption{Approximation of the $S(2,4)$ decider pipeline as implemented in \CoqBB. All the $\BBTxFTNF$ enumerated 2-state 4-symbol machines are decided by this pipeline, which solves $S(2,4) = \BBTxF$, see Theorem~\ref{th:BB2x4}. The exact pipeline, with deciders parameters is given in Appendix~\ref{app:pipelines}. }\label{tab:pipelineBB2x4}
\end{table}

\begin{table}[h!]
    \centering
    \begin{tabular}{|l|rrr|}
        \hline
        $S(4)$ decider pipeline                                                              & Nonhalt & Halt    & Total decided \\
        \hline
        1. Loops, see \textbf{Section~\ref{sec:loops}}                                       & 588,373 & 249,693 & 838,066       \\
        2. $n$-gram Closed Position Set (NGramCPS), see \textbf{Section~\ref{sec:n-gramCPS}} & 20,841  & 0       & 0             \\
        3. Repeated Word List (RepWL), see \textbf{Section~\ref{sec:RepWL}}                  & 2       & 0       & 2             \\
        \hline
        Total                                                                                & 609,216 & 249,693 & 858,909       \\
        \hline
    \end{tabular}
    \caption{Approximation of the $S(4)$ decider pipeline as implemented in \CoqBB. All the $\BBtheFourthTNF$ enumerated 4-state machines are decided by this pipeline, which brings confirmation that $S(4) = \BBtheFourth$ \cite{Brady83}, see Theorem~\ref{th:BB4}. The exact pipeline, with deciders parameters is given in Appendix~\ref{app:pipelines}. }\label{tab:pipelineBB4}
\end{table}